\newgeometry{top=2in,left=1.27in,right=2cm,bottom=0.5in}
\begin{center}	
	\textbf{\Large ACKNOWLEDGEMENT}\\
\end{center}
\vspace{2cm}
\par
We would like to express our gratitude to all those who helped us to complete this
project. We would like to thank the faculty members in the Department of Computer
Science and Engineering, SCT College of Engineering, Thiruvananthapuram, for giving
us the opportunity and facilities to undertake this project and for its successful
completion.
\par
We are deeply indebted to our Guide, \textbf{Mrs Binu Rajan}, Assistant Professor,
Department of Computer Science and Engineering, SCT College of Engineering, whose
help, stimulating suggestions and encouragement helped us throughout the duration of
the project. Her technical know how and valuable hints were of immense help to us for
carrying out the project.
\par
We are also obliged to \textbf{ Dr. Subu Surendran}, Professor and Head of Department
(Department of Computer Science and Engineering), SCT College of Engineering, for the
support and encouragement provided by him to make this project a success.

\newpage

\newgeometry{top=2in,left=1.27in,right=2cm,bottom=1in}

%%Abstract Page
\thispagestyle{empty}
\begin{center}	
	\textbf{\Large ABSTRACT}\\
\end{center}
\vspace{2cm}
\begin{comment}
\par
Social media plays a big role in our day-to day lives. Automatic identification of user interest from social media has gained much attention in the recent years. Over the last few years, micro-blogging services like Twitter have become a great source of information from friends, celebrities, organizations and a means for building social networks. In Twitter, users could post tweets about a wide range of topics. These tweets could be analyzed to identify the user’s interests, which could be used to personalize recommendations for that user.These services are being increasingly used for real-time information sharing, news and recommendations.
This report describes the creation, implementation and evaluation of a classifier, trained using supervised machine learning techniques, which takes tweets of a person as input, and classifies them into a set of predefined categories. These categories are Sports, Finance, Technology and Entertainment. The report also explains location-based tweet analysis on products or events which provide positive,negative and neutral tweets and also gives the percentage of each of it. 
It  discusses the goals , the necessary data collected to train and test the classifier.The polarity of each word is assigned and it is used in product analysis . The
output of the classifier and the product analyzer are shown with the help of an  interface. 
\hfill
\end{comment}
With the advent of social networking and micro-blogging sites enormous amount of data is being generated. Thus the analysis of the data in these social networking sites provides useful information about the users which can be used for ad-targeting, recommendation etc. The data from micro-blogging sites can be analyzed to understand the user behavior, user’s likeness etc. These analysis can then be used for ad-targeting and recommendation. Also the micro-blogging sites provide a platform to express one’s opinions on topics, events, products etc. Various sentiment analysis can be done on this data to find the user’s  sentiment.
\\
The project is divided into two, user interest identification and location based sentiment analysis. The user interest identification aims to identify the proportion of tweets the user has tweeted in the categories Sports, Finance, Politics, Technology and Entertainment. The location based sentiment aims to analyze the sentiment about a particular entity on a given location. The various tweets of users about a particular entity in a particular region are retrieved and analyzed.